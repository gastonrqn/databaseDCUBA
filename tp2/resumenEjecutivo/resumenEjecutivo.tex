\documentclass[a4paper, 10pt, twoside]{article}

\usepackage[top=1in, bottom=1in, left=1in, right=1in]{geometry}
\usepackage[utf8]{inputenc}
\usepackage[spanish, es-ucroman, es-noquoting]{babel}
\usepackage{setspace}
\usepackage{fancyhdr}
\usepackage{lastpage}
\usepackage{amsmath}
\usepackage{amsfonts}
\usepackage{amsthm}
\usepackage{verbatim}
\usepackage{graphicx}
\usepackage{float}
\usepackage{enumitem} % Provee macro \setlist
\usepackage{tabularx}
\usepackage{multirow}
\usepackage{hyperref}
\usepackage{multicol}
\usepackage[toc, page]{appendix}
\usepackage{ulem} % para subrayados especiales
\usepackage{color}
\usepackage{framed}
\usepackage{ifthen}

\definecolor{shadecolor}{rgb}{0.95,0.95,0.95}


%%%%%%%%%% Configuración de Fancyhdr - Inicio %%%%%%%%%%
\pagestyle{fancy}
\thispagestyle{fancy}
\lhead{Bases de Datos Orientadas a Grafos}
\rhead{Heredia · Requeni · Vega · Vita}
\renewcommand{\footrulewidth}{0.4pt}
\cfoot{\thepage /\pageref{LastPage}}

\fancypagestyle{caratula} {
   \fancyhf{}
   %\cfoot{\thepage /\pageref{LastPage}}
   \renewcommand{\headrulewidth}{0pt}
   \renewcommand{\footrulewidth}{0pt}
}
%%%%%%%%%% Configuración de Fancyhdr - Fin %%%%%%%%%%


%%%%%%%%%% Configuración de Hyperref - Inicio %%%%%%%%%%
\hypersetup{%
 % Para que el PDF se abra a pagina completa.
  pdfstartview= {FitH \hypercalcbp{\paperheight-\topmargin-1in-\headheight}},
  pdfauthor={Heredia,Requeni,Vega,Vita},
  pdfsubject={BD-TP2},
 %pdfkeywords={keyword1} {key2} {key3},
 colorlinks=true,
  linkcolor=black,
  urlcolor=blue
}
%%%%%%%%%% Configuración de Hyperref - Fin %%%%%%%%%%


%%%%%%%%%% Miscelánea - Inicio %%%%%%%%%%
% Evita que el documento se estire verticalmente para ocupar el espacio vacío
% en cada página.
\raggedbottom

% Deshabilita sangría en la primer línea de un párrafo.
\setlength{\parindent}{0em}

% Separación entre párrafos.
\setlength{\parskip}{0.5em}

% Separación entre elementos de listas.
\setlist{itemsep=0.5em}

% Asigna la traducción de la palabra 'Appendices'.
\renewcommand{\appendixtocname}{Apéndices}
\renewcommand{\appendixpagename}{Apéndices}
%%%%%%%%%% Miscelánea - Fin %%%%%%%%%%


\begin{document}
\normalem

\pagenumbering{arabic} 

%%%%%%%%%%%%%%%%%%%%%%%%%%%%%%%%%%%%%%%%%%%%%%%%%%%%%%%%%%%%%%%%%%%%%%%%%%%%%%%
%% Encabezado                                                                %%
%%%%%%%%%%%%%%%%%%%%%%%%%%%%%%%%%%%%%%%%%%%%%%%%%%%%%%%%%%%%%%%%%%%%%%%%%%%%%%%

\begin{center}
\begin{Large}
\underline{Bases de Datos Orientadas a Grafos}
\end{Large}

\underline{Heredia, Requeni, Vega, Vita}
 
\end{center}



\section{Introducción}

Una base de datos orientada a grafos es un modelo de base de datos caracterizado por:
\begin{itemize}
 \item \emph{Los datos y el esquema se modelan con grafos}
 \item \emph{La manipulación de datos se expresa usando transformaciones de grafos}
\end{itemize}

Sobre estos modelos se pueden definir también restricciones que aseguren integridad y consistencia de datos, como puede ser la consistencia
entre la instancia y el esquema, dependencias funcionales, dominio de las propiedades, etc.

\section{Utilidad}

Las bases de datos relacionales modelan a las entidades utilizando tablas. Las relaciones entre entidades se modelan usando campos
que identifican registros en otra tabla (Primary Keys y Foreign Keys) y es necesario introducir tablas intermedias para las
interrelaciones ``muchos a muchos''. Para materializar una interrelación entre dos o más entidades, se introduce el concepto de Join (o junta) que
mediante algoritmos muy costosos recorren las tablas involucradas uniendo los registros que se interrelacionan.

Cuando tenemos un modelo en el que las relaciones son más importantes que las entidades y sus atributos, 
las bases de datos relacionales pierden su utilidad, dado que la operación más utilizada pasa a ser el Join, que es a la vez
la operación más costosa. Esto es porque las bases de datos relacionales están optimizadas para almacenar muchas tablas con muchos registros (muchas
entidades con muchos atributos) y pocas interrelaciones. Cuantas más interrelaciones y tablas de interrelación aparecen, mayor es la cantidad de Joins
que se necesitan para hacer consultas básicas, y en tal caso la complejidad temporal de las consultas aumenta exponencialmente.

Es en este contexto que ganan terreno las bases de datos orientadas a grafos, donde las relaciones son los \emph{ciudadanos de primera clase}.
A diferencia de las relacionales, las orientadas a grafos optimizan y flexibilizan el modelado de las relaciones. La información que un Join genera
en una base relacional es la misma información que el grafo almacena todo el tiempo usando ejes para unir nodos (que desde el punto de vista de un join,
serían los registros que se unen).

A continuación presentaremos un ejemplo y luego analizaremos en qué casos es conveniente usar una base de datos orientada a grafos.

Supongamos que tenemos empleados y departamentos. Cada empleado puede trabajar en más de un departamento y en un departamento pueden trabajar
muchos empleados. La base de datos relacional quedaría así:

\begin{center}
\includegraphics[width=0.5\textwidth]{../informe/img/ej1_relacional.png}
\end{center}

Este mismo modelo implementado en una base de datos orientada a grafos, quedaría así:

\begin{center}
\includegraphics[width=0.5\textwidth]{../informe/img/ej1_grafo.png}
\end{center}

Observemos que no fue necesario usar estructuras adicionales como la tabla {\tt Dept\_Member} para modelar las relaciones. Simplemente tendremos
un nodo por cada persona y un nodo por cada departamento, y un eje por cada registro que antes había en la tabla {\tt Dept\_Member}. En este caso
la complejidad espacial es similar, pero la complejidad temporal de hacer un Join es mucho mayor que simplemente acceder a un nodo persona y navegar a 
sus vecinos de tipo {\tt Department} (que se puede hacer muy rápidamente).

Por otro lado, cuando tenemos una entidad que se relaciona consigo misma, se crea una relación recursiva o jerárquica. Por ejemplo la relación
empleado - supervisor (el supervisor es un empleado). Hacer consultas a lo largo de estas jerarquías en bases relacionales es increíblemente
costoso, mientras que en un grafo
simplemente tenemos que buscar caminos de distintas longitudes para acceder a todos los ancestros (y esto se puede expresar fácilmente y resolver
rápidamente).

\section{Conclusión}
Las bases de datos orientadas a grafos son útiles cuando las interrelaciones de entidades predominan, o dicho de
otra manera, cuando es importante la interconectividad o la topología de los datos. En estos casos mejoran considerablemente la complejidad
temporal de las consultas y permiten un análisis más intuitivo y exaustivo.

Los usos típicos de estas bases de datos son las redes ferroviarias,
redes sociales, datos geográficos y espaciales, genómica, biología, sistemas de recomendación, entre otros.

\end{document}

