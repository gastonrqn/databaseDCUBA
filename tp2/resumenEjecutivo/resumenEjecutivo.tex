\documentclass[a4paper, 10pt, twoside]{article}

\usepackage[top=1in, bottom=1in, left=1in, right=1in]{geometry}
\usepackage[utf8]{inputenc}
\usepackage[spanish, es-ucroman, es-noquoting]{babel}
\usepackage{setspace}
\usepackage{fancyhdr}
\usepackage{lastpage}
\usepackage{amsmath}
\usepackage{amsfonts}
\usepackage{amsthm}
\usepackage{verbatim}
\usepackage{graphicx}
\usepackage{float}
\usepackage{enumitem} % Provee macro \setlist
\usepackage{tabularx}
\usepackage{multirow}
\usepackage{hyperref}
\usepackage{multicol}
\usepackage[toc, page]{appendix}
\usepackage{ulem} % para subrayados especiales
\usepackage{color}
\usepackage{framed}
\usepackage{ifthen}

\definecolor{shadecolor}{rgb}{0.95,0.95,0.95}


%%%%%%%%%% Configuración de Fancyhdr - Inicio %%%%%%%%%%
\pagestyle{fancy}
\thispagestyle{fancy}
\lhead{Bases de Datos Orientadas a Grafos}
\rhead{Heredia · Requeni · Vega · Vita}
\renewcommand{\footrulewidth}{0.4pt}
\cfoot{\thepage /\pageref{LastPage}}

\fancypagestyle{caratula} {
   \fancyhf{}
   %\cfoot{\thepage /\pageref{LastPage}}
   \renewcommand{\headrulewidth}{0pt}
   \renewcommand{\footrulewidth}{0pt}
}
%%%%%%%%%% Configuración de Fancyhdr - Fin %%%%%%%%%%


%%%%%%%%%% Configuración de Hyperref - Inicio %%%%%%%%%%
\hypersetup{%
 % Para que el PDF se abra a pagina completa.
  pdfstartview= {FitH \hypercalcbp{\paperheight-\topmargin-1in-\headheight}},
  pdfauthor={Heredia,Requeni,Vega,Vita},
  pdfsubject={BD-TP2},
 %pdfkeywords={keyword1} {key2} {key3},
 colorlinks=true,
  linkcolor=black,
  urlcolor=blue
}
%%%%%%%%%% Configuración de Hyperref - Fin %%%%%%%%%%


%%%%%%%%%% Miscelánea - Inicio %%%%%%%%%%
% Evita que el documento se estire verticalmente para ocupar el espacio vacío
% en cada página.
\raggedbottom

% Deshabilita sangría en la primer línea de un párrafo.
\setlength{\parindent}{0em}

% Separación entre párrafos.
\setlength{\parskip}{0.5em}

% Separación entre elementos de listas.
\setlist{itemsep=0.5em}

% Asigna la traducción de la palabra 'Appendices'.
\renewcommand{\appendixtocname}{Apéndices}
\renewcommand{\appendixpagename}{Apéndices}
%%%%%%%%%% Miscelánea - Fin %%%%%%%%%%


\begin{document}
\normalem

\pagenumbering{arabic} 

%%%%%%%%%%%%%%%%%%%%%%%%%%%%%%%%%%%%%%%%%%%%%%%%%%%%%%%%%%%%%%%%%%%%%%%%%%%%%%%
%% Encabezado                                                                %%
%%%%%%%%%%%%%%%%%%%%%%%%%%%%%%%%%%%%%%%%%%%%%%%%%%%%%%%%%%%%%%%%%%%%%%%%%%%%%%%

\begin{center}
\begin{Large}
\underline{Bases de Datos Orientadas a Grafos}
\end{Large}

\underline{Heredia, Requeni, Vega, Vita}
 
\end{center}



\section{Introducción}

Una base de datos orientada a grafos es un modelo de base de datos caracterizado por:
\begin{itemize}
 \item \emph{Los datos y el esquema se modelan con grafos}: pueden ser grafos simples, dirigidos, con o sin etiquetas, o incluso pueden ser
 generalizaciones de grafos que utilicen hipernodos (los nodos contienen grafos) o hiperejes (ejes que unen más de un nodo).
 \item \emph{La manipulación de datos se expresa usando transformaciones de grafos}: Para crear, modificar y consultar los datos
 de la base se utilizan conceptos relacionados a grafos como caminos, vecinos, subgrafos, conectividad, diámetro, etc. Para expresar estas
 estructuras en un lenguaje de manipulación de datos se suele utilizar pattern matching.
\end{itemize}

Sobre estos modelos se pueden definir también restricciones que aseguren integridad y consistencia de datos, como puede ser la consistencia
entre la instancia y el esquema, dependencias funcionales, dominio de las propiedades, etc.

\section{Motivación}

El modelo de entidad relación plantea estructurar a los datos en entidades e interrelaciones entre entidades. Este modelo se puede pensar
como un grafo donde cada entidad es un nodo y las interrelaciones son ejes entre ellos. Tanto los nodos como los ejes tienen labels y podrían
tener atributos.

Las bases de datos relacionales modelan a las entidades utilizando tablas. Las interrelaciones se modelan usando campos
que identifican registros en otra tabla (Primary Keys y Foreign Keys) y es necesario introducir tablas intermedias para las
interrelaciones ``muchos a muchos''. Para materializar una interrelación entre dos o más entidades, se introduce el concepto de Join (o junta) que
mediante algoritmos muy costosos recorren las tablas involucradas uniendo los registros que se interrelacionan.

Decimos que las relaciones son los \emph{ciudadanos de primera clase} de un modelo de datos cuando su importancia es mayor que la de las entidades y sus
atributos. En tal escenario, las bases de datos relacionales pierden su utilidad, dado que la operación más utilizada pasa a ser el Join, que es a la vez
la operación más costosa. Esto es porque las bases de datos relacionales están optimizadas para almacenar muchas tablas con muchos registros (muchas
entidades con muchos atributos) y pocas interrelaciones. Cuantas más interrelaciones y tablas de interrelación aparecen, mayor es la cantidad de Joins
que se necesitan para hacer consultas básicas, y en tal caso la complejidad temporal de las consultas aumenta exponencialmente.

Es en este contexto que ganan terreno las bases de datos orientadas a grafos, donde las relaciones son los \emph{ciudadanos de primera clase}.
A diferencia de las relacionales, las orientadas a grafos optimizan y flexibilizan el modelado de las relaciones. La información que un Join genera
en una base relacional es la misma información que el grafo almacena todo el tiempo usando ejes para unir nodos (que desde el punto de vista de un join,
serían los registros que se unen).

A continuación presentaremos un ejemplo y luego analizaremos en qué casos es conveniente usar una base de datos orientada a grafos.

Supongamos que tenemos empleados y departamentos. Cada empleado puede trabajar en más de un departamento y en un departamento pueden trabajar
muchos empleados. La base de datos relacional quedaría así:

\begin{center}
\includegraphics[width=0.5\textwidth]{../informe/img/ej1_relacional.png}
\end{center}

Este mismo modelo implementado en una base de datos orientada a grafos, quedaría así:

\begin{center}
\includegraphics[width=0.5\textwidth]{../informe/img/ej1_grafo.png}
\end{center}

Observemos que no fue necesario usar estructuras adicionales como la tabla {\tt Dept\_Member} para modelar las relaciones. Simplemente tendremos
un nodo por cada persona y un nodo por cada departamento, y un eje por cada registro que antes había en la tabla {\tt Dept\_Member}. En este caso
la complejidad espacial es similar, pero la complejidad temporal de hacer un Join es mucho mayor que simplemente acceder a un nodo persona y navegar a 
sus vecinos de tipo {\tt Department} en $O(1)$. A esto nos referíamos cuando dijimos que en las bases de datos orientadas a grafos, el join ya está hecho,
es parte de la implementación de la base misma.

Veamos otro ejemplo: Tenemos una única entidad empleado ({\tt EMPLOYEE}) con algunos atributos básicos y la relación ``reporta a'' ({\tt REPORTS\_TO}).

\begin{center}
\includegraphics[width=0.4\textwidth]{../informe/img/ej2_der.png}
\end{center}

La implementación relacional consiste de una única tabla empleado con los mismo atributos y una columna extra para la FK al supervisor.

\begin{center}
\includegraphics[width=0.2\textwidth]{../informe/img/ej2_relacional.png}
\end{center}

Supongamos que ``Andrew'' es el nombre del jefe y queremos obtener todos los empleados que reportan directamente al jefe y por cada uno de ellos
la cantidad de empleados que reportan directamente a ellos. Vamos a suponer que ``reportar directamente'' puede ser una jerarquía de hasta 3 niveles.
Por ejemplo, la siguiente instancia:

\begin{center}
\includegraphics[width=\textwidth]{../informe/img/ej2_ejemploResultado.png}
\end{center}

Debería retornar:
\begin{center}
\begin{tabular}{| c | c |}
\hline
 {\bf Subordinate} & {\bf Total}\\
\hline
 Andrew & 3\\
\hline
 Bob & 3\\
\hline
 Alice & 2\\
\hline
 John & 1 \\
\hline
\end{tabular}
\end{center}

Para resolver esto en SQL (sin usar recursión, que no siempre está disponible) deberíamos usar una consulta similar a esta (difieren los nombres de tabla
y atributos):

\begin{center}
\includegraphics[width=0.7\textwidth]{../informe/img/ej2_sql.png}
\end{center}

Sin embargo, si usamos una base de datos orientada a grafos, usando el lenguaje Cypher, la consulta se simplifica considerablemente:

\begin{center}
\includegraphics[width=0.4\textwidth]{../informe/img/ej2_cypher.png}
\end{center}

Las bases de datos relacionales soportan las relaciones recursivas mediante joins recursivos y usando selects anidados, mientras que en un grafo
simplemente tenemos que buscar caminos de distintas longitudes para acceder a todos los ancestros.

Desde un punto de vista general, las bases de datos orientadas a grafos son útiles cuando las interrelaciones de entidades predominan, o dicho de
otra manera, cuando es importante la interconectividad o la topología de los datos. Los usos típicos de estas bases de datos son las redes ferroviarias,
redes sociales, datos geográficos y espaciales, genómica, biología, sistemas de recomendación, entre otros.

Observar que el DER que ya conocemos sirve como esquema para este tipo de bases de datos, no es necesario una conversión a tabla, aunque por otro
lado también es importante observar que estas bases de datos podrían no tener esquema (permiten agregar interrelaciones y
nuevas entidades de forma dinámica).

\section{Modelo Teórico}

Para estudiar el modelo teórico de las bases de datos orientadas a grafos, estudiaremos tres focos distintos: Las entidades, las interrelaciones
y la integridad de los datos.

\begin{itemize}
 \item {\bf Entidades:} Son los nodos del grafo. Las entidades del esquema definen los ``tipos de datos'', especificando qué atributos
 deben (o pueden) tener y de qué tipos, así como también con qué otras entidades se pueden relacionar. Por otro lado, las entidades de la instancia
 son los nodos con los atributos y relaciones concretas, y tienen un id y un tipo (asociado al esquema).
\end{itemize}


\end{document}

