Entre los modelos db-grafos, hay un trabajo sustancial centrado en lenguajes de consulta, el problema de la consultas a grafos, la presentación visual de los resultados, y los lenguajes gráficos de consulta. Debido a esto, hemos optado por realizar un estudio sobre el mismo, limitandonos a describir los lenguajes de consulta gráfica más importantes.\newline
Los lenguajes de consulta G, G +, y GraphLog integran una familia de lenguajes gráficos relacionados definidas sobre un simple modelo de grafo.
\begin{itemize}
\item El lenguaje de consulta gráfica G [1987] se basa en expresiones regulares que permiten la simple formulación de consultas recursivas. Un gráfico de consultas en G es un conjunto de multigrafos dirigidos etiquetados donde los nodos pueden ser variables o constantes, y las aristas se puede marcar con expresiones regulares. El resultado de una consulta es la unión de todos los grafos de consulta que relacionan subgrafos de la instancia.
\item G se convirtió en un lenguaje más potente llamado G +, en la que un gráfico de consulta permanece como el bloque de construcción básico. Una simple consulta en G + tiene dos elementos, un grafo de consulta que especifica la clase de los patrones de búsqueda, y un grafo de resumen, lo que representa la forma de reestructurar la respuesta obtenida por el grafo de la consulta.
\item GraphLog [1989] es un lenguaje de consulta de hipertexto que se extiende G + agregando la negación y la unificación del concepto de un grafo de consulta. Una consulta es ahora un solo patrón grafo que contiene una arista distinguida, que se corresponde con la arista de la reestructuración del grafo de resumen en G +. El efecto de la consulta es encontrar todas las instancias del patrón que se producen en el grafo de la base de datos y para cada uno de ellos, para definir un enlace virtual representada por la arista distinguida. GraphLog incluye un operador clausura transitiva implícita, que sustituye al mecanismo habitual de recursión.
\end{itemize}

La propuesta del G-Log [1995], incluye un lenguaje declarativo para objetos complejos con identidad. Utiliza la noción lógica de satisfacción de reglas para evaluar las consultas que se expresan como los programas G-Log. G-Log programs son conjuntos de reglas de la base de grafo, que especifican cómo van a cambiar el esquema y la instancia de la base de datos. G-Log es una base de grafo, declarativo, no deterministico, computacionalmente completo que no sufre el problema de copia eliminación. G-Log es un buen ejemplo de lenguaje gráfico de consulta (ver Fig).\newline
En el contexto de modelar grafos orientados, hay lenguajes de consulta que consideran transformaciones de bases de datos como las transformaciones de grafos (que pueden ser interpretadas como consultas de bases de datos y updates). Se basan en la coincidencia de patrones del grafo y permiten al usuario especificar los insertions y deletions de nodo de una manera gráfica. Además, GOAL incluye la noción de puntos fijos con el fin de manejar la recursión derivado de una lista finita de insertions y deletions. Pamal propuso la inclusión del loop y de procedimientos y programas de construcciones. Debemos tener en cuenta que los formalismos de manipulación grafos orientados sobre la base de patrones permiten una manera dirigida por la sintaxis de trabajar mucho más natural que las interfaces basadas en texto.\newline
GROOVY [1991] introduce un lenguaje de manipulación hyper grafo (HML) para consultar y actualizar hyper grafos etiquetados. Define dos operadores básicos para consultar hyper grafos por identificador o por valor, y ocho operadores para la manipulación (insertions y deletinos) de hyper grafos e hyper aristas.
Watters y Pastor [1990] presentan un framework para el acceso de datos generales sobre la base de hyper grafos que incluyen operadores de creación de aristas y los operadores establecidos como intersección, unión, y la diferencia. En un contexto diferente, Tompa [1989] introduce operaciones básicas sobre estructuras hyper grafos que representan el estado del usuario y visitas en páginas orientadas a datos de hiper texto.\newline
También hay propuestas de lenguajes de consulta que tienen que ver con las estructuras hyper nodos:
\begin{itemize}
\item Levene y Poulovassilis [1990] definen una consulta y actualización de lenguaje basado en la lógica, donde las consultas se expresan como conjuntos de reglas hyper nodos (H-reglas) que se llaman programas hyper nodos. El lenguaje de consulta define un operador que infiere nuevos hypernodes de la instancia, mediante el conjunto de reglas en un programa hyper nodo.
\item Este Lenguaje de consulta se extendió por HyperLOG, incluyendo deletions así como insertions, y discutir con más detalle las cuestiones de aplicación. Una capacidad de Turing-máquina completa se obtiene sumando la composición, las construcciones condicionales, y la iteración. HyperLOG es computacionalmente completo, aunque la evaluación de los programas HyperLOG es intratable en el caso general.
\item Además, HNQL [1995] define un conjunto de operadores para consulta declarativa y update de hyper nodos. También incluye la asignación, la composición secuencial, condicional (para hacer inferencias), for loop, y while loop constructs.
\end{itemize}

Glide [2002] es un lenguaje de consulta de grafo donde las consultas se expresan utilizando una notación lineal formado por las etiquetas y expresiones regulares. Glide utiliza un método llamado GraphGrep basado en relacionar subgrafo para responder a las consultas.\newline
Flesca y Greco [1999] muestran cómo utilizar lenguajes parcialmente ordenados para definir consultas de camino para buscar en bases de datos y presentan resultados sobre su complejidad.\newline
Cardelli [2002] introduce una lógica espacial para el razonamiento sobre los grafos, y define un lenguaje de consulta basado en pattern matching y la recursión. Esta lógica de grafo combina la lógica de primer orden con la agregación de conectivos estructurales. Una consulta pide una sustitucion de variables tal que una relación de satisfacción determina que grafo satisface a las fórmulas. El lenguaje de consulta se basa en las consultas que construyen nuevos grafos de viejos, y transductores que relacionan grafos de entrada con grafos de salida.\newline
El modelo Simatic-XT [1994] define un lenguaje de consulta. Incluye operadores básicos que tienen que ver con los datos encapsulados (anidación de hyper nodos), operadores establecidos (unión, concatenación, selección y diferencia), y los operadores de alto nivel (caminos, la inclusión e intersecciones).\newline
WEB [1993;1994] es un lenguaje de programación declarativa basada en una lógica de grafo y orientado a la consulta de datos del genoma. Programas WEB definen las plantillas de grafos para la creación, manipulación y consulta de los objetos y las relaciones en la base de datos. Estas operaciones son contestadas relacionando grafos en instancias válidas.\newline
Gram [1992] presenta un álgebra de consulta (con estilo SQL) donde se usan expresiones regulares más tipos de datos para seleccionar el camino en un grafo. Se utiliza un modelo de datos donde los caminos son objetos básicos. Una expresión de camino es una expresión regular sin unión, cuyo lenguaje contiene sólo secuencias que alterna entre tipos de nodos y aristas, comenzando y terminando con un tipo de nodo. El lenguaje de consulta se basa en un álgebra hyper camino con las operaciones cerradas bajo el conjunto de hyper caminos.\newline
GraphDB [1994] incluye una clase de objetos llamados clase trayectoria, que se utilizan para representar varios caminos en la base de datos.

Uno de los problemas más fundamentales de grafos en lenguajes de consulta es para computar la accesibilidad de la información, lo que se traduce en problemas de caminos caracterizados y expresadas por consultas recursivas. Encontrar caminos simples con propiedades deseadas en grafos directos es difícil, y esencialmente cada propiedad no trivial da origen a un problema NP-completo.

\begin{center}
\includegraphics[width=0.8\textwidth]{img/lenguaje-consulta.png}
\end{center}