\documentclass[a4paper, 10pt, twoside]{article}

\usepackage[top=1in, bottom=1in, left=1in, right=1in]{geometry}
\usepackage[utf8]{inputenc}
\usepackage[spanish, es-ucroman, es-noquoting]{babel}
\usepackage{setspace}
\usepackage{fancyhdr}
\usepackage{lastpage}
\usepackage{amsmath}
\usepackage{amsfonts}
\usepackage{amsthm}
\usepackage{verbatim}
\usepackage{graphicx}
\usepackage{float}
\usepackage{enumitem} % Provee macro \setlist
\usepackage{tabularx}
\usepackage{multirow}
\usepackage{hyperref}
\usepackage{multicol}
\usepackage[toc, page]{appendix}
\usepackage{ulem} % para subrayados especiales
\usepackage{color}
\usepackage{framed}
\usepackage{ifthen}

\definecolor{shadecolor}{rgb}{0.95,0.95,0.95}


%%%%%%%%%% Configuración de Fancyhdr - Inicio %%%%%%%%%%
\pagestyle{fancy}
\thispagestyle{fancy}
\lhead{Trabajo Práctico 2 · Bases de Datos}
\rhead{Heredia · Requeni · Vega · Vita}
\renewcommand{\footrulewidth}{0.4pt}
\cfoot{\thepage /\pageref{LastPage}}

\fancypagestyle{caratula} {
   \fancyhf{}
   %\cfoot{\thepage /\pageref{LastPage}}
   \renewcommand{\headrulewidth}{0pt}
   \renewcommand{\footrulewidth}{0pt}
}
%%%%%%%%%% Configuración de Fancyhdr - Fin %%%%%%%%%%


%%%%%%%%%% Configuración de Hyperref - Inicio %%%%%%%%%%
\hypersetup{%
 % Para que el PDF se abra a pagina completa.
  pdfstartview= {FitH \hypercalcbp{\paperheight-\topmargin-1in-\headheight}},
  pdfauthor={Heredia,Requeni,Vega,Vita},
  pdfsubject={BD-TP2},
 %pdfkeywords={keyword1} {key2} {key3},
 colorlinks=true,
  linkcolor=black,
  urlcolor=blue
}
%%%%%%%%%% Configuración de Hyperref - Fin %%%%%%%%%%


%%%%%%%%%% Miscelánea - Inicio %%%%%%%%%%
% Evita que el documento se estire verticalmente para ocupar el espacio vacío
% en cada página.
\raggedbottom

% Deshabilita sangría en la primer línea de un párrafo.
\setlength{\parindent}{0em}

% Separación entre párrafos.
\setlength{\parskip}{0.5em}

% Separación entre elementos de listas.
\setlist{itemsep=0.5em}

% Asigna la traducción de la palabra 'Appendices'.
\renewcommand{\appendixtocname}{Apéndices}
\renewcommand{\appendixpagename}{Apéndices}
%%%%%%%%%% Miscelánea - Fin %%%%%%%%%%


\begin{document}
\pagenumbering{gobble} 
 \normalem


%%%%%%%%%%%%%%%%%%%%%%%%%%%%%%%%%%%%%%%%%%%%%%%%%%%%%%%%%%%%%%%%%%%%%%%%%%%%%%%
%% Carátula                                                                  %%
%%%%%%%%%%%%%%%%%%%%%%%%%%%%%%%%%%%%%%%%%%%%%%%%%%%%%%%%%%%%%%%%%%%%%%%%%%%%%%%


\thispagestyle{caratula}

\begin{center}

\includegraphics[height=2cm]{caratula/DC.png} 
\hfill
\includegraphics[height=2cm]{caratula/UBA.jpg} 

\vspace{2cm}

Departamento de Computación,\\
Facultad de Ciencias Exactas y Naturales,\\
Universidad de Buenos Aires

\vspace{4cm}

\begin{Huge}
Trabajo Práctico 2
\end{Huge}

\vspace{0.5cm}

\begin{huge}
Bases de Datos Orientadas a Grafos
\end{huge}

\vspace{0.5cm}

\begin{Large}
Bases de Datos
\end{Large}

\vspace{1cm}

Primer Cuatrimestre de 2015

\vspace{3.5cm}

\begin{Large}
Grupo 4
\end{Large}

\vspace{0.5cm}

\begin{tabular}{|c|c|c|}
\hline
Apellido y Nombre & LU & E-mail\\
\hline
Heredia, Martín  & 146/11 & {\tt martin.herediaf@gmail.com}\\
Requeni, Gastón  & 400/11 & {\tt grequeni@hotmail.com}\\
Vega, Leandro    & 698/11 & {\tt leandrogvega@gmail.com}\\
Vita, Sebastián  & 149/11 & {\tt sebastian\_vita@yahoo.com.ar}\\
\hline
\end{tabular}

\end{center}

\newpage
\pagenumbering{arabic} 

%%%%%%%%%%%%%%%%%%%%%%%%%%%%%%%%%%%%%%%%%%%%%%%%%%%%%%%%%%%%%%%%%%%%%%%%%%%%%%%
%% Índice                                                                    %%
%%%%%%%%%%%%%%%%%%%%%%%%%%%%%%%%%%%%%%%%%%%%%%%%%%%%%%%%%%%%%%%%%%%%%%%%%%%%%%%


\tableofcontents

\newpage

\section{Introducción}

Una base de datos orientada a grafos es un modelo de base de datos caracterizado por:
\begin{itemize}
 \item \emph{Los datos y el esquema se modelan con grafos}: pueden ser grafos simples, dirigidos, con o sin etiquetas, o incluso pueden ser
 generalizaciones de grafos que utilicen hipernodos (los nodos contienen grafos) o hiperejes (ejes que unen más de un nodo).
 \item \emph{La manipulación de datos se expresa usando transformaciones de grafos}: Para crear, modificar y consultar los datos
 de la base se utilizan conceptos relacionados a grafos como caminos, vecinos, subgrafos, conectividad, diámetro, etc. Para expresar estas
 estructuras en un lenguaje de manipulación de datos se suele utilizar pattern matching.
\end{itemize}

Sobre estos modelos se pueden definir también restricciones que aseguren integridad y consistencia de datos, como puede ser la consistencia
entre la instancia y el esquema, dependencias funcionales, dominio de las propiedades, etc.


\newpage

\section{Motivación}

El modelo de entidad relación plantea estructurar a los datos en entidades e interrelaciones entre entidades. Este modelo se puede pensar
como un grafo donde cada entidad es un nodo y las interrelaciones son ejes entre ellos. Tanto los nodos como los ejes tienen labels y podrían
tener atributos.

Las bases de datos relacionales modelan a las entidades utilizando tablas. Las interrelaciones se modelan usando campos
que identifican registros en otra tabla (Primary Keys y Foreign Keys) y es necesario introducir tablas intermedias para las
interrelaciones ``muchos a muchos''. Para materializar una interrelación entre dos o más entidades, se introduce el concepto de Join (o junta) que
mediante algoritmos muy costosos recorren las tablas involucradas uniendo los registros que se interrelacionan.

Decimos que las relaciones son los \emph{ciudadanos de primera clase} de un modelo de datos cuando su importancia es mayor que la de las entidades y sus
atributos. En tal escenario, las bases de datos relacionales pierden su utilidad, dado que la operación más utilizada pasa a ser el Join, que es a la vez
la operación más costosa. Esto es porque las bases de datos relacionales están optimizadas para almacenar muchas tablas con muchos registros (muchas
entidades con muchos atributos) y pocas interrelaciones. Cuantas más interrelaciones y tablas de interrelación aparecen, mayor es la cantidad de Joins
que se necesitan para hacer consultas básicas, y en tal caso la complejidad temporal de las consultas aumenta exponencialmente.

Es en este contexto que ganan terreno las bases de datos orientadas a grafos, donde las relaciones son los \emph{ciudadanos de primera clase}.
A diferencia de las relacionales, las orientadas a grafos optimizan y flexibilizan el modelado de las relaciones. La información que un Join genera
en una base relacional es la misma información que el grafo almacena todo el tiempo usando ejes para unir nodos (que desde el punto de vista de un join,
serían los registros que se unen).

A continuación presentaremos un ejemplo y luego analizaremos en qué casos es conveniente usar una base de datos orientada a grafos.

Supongamos que tenemos empleados y departamentos. Cada empleado puede trabajar en más de un departamento y en un departamento pueden trabajar
muchos empleados. La base de datos relacional quedaría así:

\begin{center}
\includegraphics[width=0.5\textwidth]{img/ej1_relacional.png}
\end{center}

Este mismo modelo implementado en una base de datos orientada a grafos, quedaría así:

\begin{center}
\includegraphics[width=0.5\textwidth]{img/ej1_grafo.png}
\end{center}

Observemos que no fue necesario usar estructuras adicionales como la tabla {\tt Dept\_Member} para modelar las relaciones. Simplemente tendremos
un nodo por cada persona y un nodo por cada departamento, y un eje por cada registro que antes había en la tabla {\tt Dept\_Member}. En este caso
la complejidad espacial es similar, pero la complejidad temporal de hacer un Join es mucho mayor que simplemente acceder a un nodo persona y navegar a 
sus vecinos de tipo {\tt Department} en $O(1)$. A esto nos referíamos cuando dijimos que en las bases de datos orientadas a grafos, el join ya está hecho,
es parte de la implementación de la base misma.

Veamos otro ejemplo: Tenemos una única entidad empleado ({\tt EMPLOYEE}) con algunos atributos básicos y la relación ``reporta a'' ({\tt REPORTS\_TO}).

\begin{center}
\includegraphics[width=0.4\textwidth]{img/ej2_der.png}
\end{center}

La implementación relacional consiste de una única tabla empleado con los mismo atributos y una columna extra para la FK al supervisor.

\begin{center}
\includegraphics[width=0.2\textwidth]{img/ej2_relacional.png}
\end{center}

Supongamos que ``Andrew'' es el nombre del jefe y queremos obtener todos los empleados que reportan directamente al jefe y por cada uno de ellos
la cantidad de empleados que reportan directamente a ellos. Vamos a suponer que ``reportar directamente'' puede ser una jerarquía de hasta 3 niveles.
Por ejemplo, la siguiente instancia:

\begin{center}
\includegraphics[width=\textwidth]{img/ej2_ejemploResultado.png}
\end{center}

Debería retornar:
\begin{center}
\begin{tabular}{| c | c |}
\hline
 {\bf Subordinate} & {\bf Total}\\
\hline
 Andrew & 3\\
\hline
 Bob & 3\\
\hline
 Alice & 2\\
\hline
 John & 1 \\
\hline
\end{tabular}
\end{center}

Para resolver esto en SQL (sin usar recursión, que no siempre está disponible) deberíamos usar una consulta similar a esta (difieren los nombres de tabla
y atributos):

\begin{center}
\includegraphics[width=0.7\textwidth]{img/ej2_sql.png}
\end{center}

Sin embargo, si usamos una base de datos orientada a grafos, usando el lenguaje Cypher, la consulta se simplifica considerablemente:

\begin{center}
\includegraphics[width=0.4\textwidth]{img/ej2_cypher.png}
\end{center}

Las bases de datos relacionales soportan las relaciones recursivas mediante joins recursivos y usando selects anidados, mientras que en un grafo
simplemente tenemos que buscar caminos de distintas longitudes para acceder a todos los ancestros.

Desde un punto de vista general, las bases de datos orientadas a grafos son útiles cuando las interrelaciones de entidades predominan, o dicho de
otra manera, cuando es importante la interconectividad o la topología de los datos. Los usos típicos de estas bases de datos son las redes ferroviarias,
redes sociales, datos geográficos y espaciales, genómica, biología, sistemas de recomendación, entre otros.

Observar que el DER que ya conocemos sirve como esquema para este tipo de bases de datos, no es necesario una conversión a tabla, aunque por otro
lado también es importante observar que estas bases de datos podrían no tener esquema (permiten agregar interrelaciones y
nuevas entidades de forma dinámica).

\newpage

\section{Modelo Teórico}

Para estudiar el modelo teórico de las bases de datos orientadas a grafos, estudiaremos tres focos distintos: Las entidades, las interrelaciones
y la integridad de los datos.

\subsection{Entidades}
Una entidad en una base de datos orientada a grafos es un nodo y se define en dos contextos distintos:
\begin{itemize}
 \item Entidad del esquema: Define el \emph{tipo} asociado a una entidad, declarando atributos y el tipo de los mismos y todas las posibles
 interrelaciones que podría tener. Los atributos pueden representarse dentro de la estructura del nodo o bien como \emph{entidades primitivas}
 con las que la entidad se relaciona.
 \begin{center}
  \includegraphics[width=0.6\textwidth]{img/ej3_entidadEsquema.png}
  \end{center}
  Cabe aclarar que no son las únicas formas de definir una entidad en un esquema pero presentamos las más representativas. Recordemos que hay
  implementaciones que utilizan hipernodos o hiperejes y allí las entidades también se definen utilizando ese tipo de grafos.
 
 \item Entidades de la instancia: Son entidades concretas, que tienen un identificador y un tipo (asociado al esquema). Además tienen atributos
 concretos y relaciones (con identificador y tipo) con otras entidades instanciadas. 
 \begin{center}
  \includegraphics[width=0.6\textwidth]{img/ej3_entidadInstancia.png}
  \end{center}
  Nuevamente aclaramos que podría haber otras formas de representación, según el tipo de grafo que se utilice.
\end{itemize}

Independientemente de la implementación, el uso de grafos para definir entidades permite crear tipos de datos no tradicionales, ya que no se está
limitado a la estructura tabular de las bases relacionales. En este sentido los grafos más simples permiten definir esquemas similares a jerarquías
de clases, objetos y colaboradores. Si a esto le sumamos hipernodos, podemos construir entidades que contengan sus propios grafos para representar la
estructura interna, dando aún más flexibilidad para la creación de estructuras complejas.

\subsection{Relaciones}

\begin{frame}[t]
\frametitle{Aspectos de una base de datos orientada a grafos}
\begin{itemize}
 \item {\bf Relaciones}
 \pause
 \begin{itemize}
  \item Grandes rasgos
 	\begin{itemize}
  	 \item Relaciones simples
  	 \pause
  	 \item Relaciones complejas
 	\end{itemize}
  \pause
  \item M\'as comunes
 	\begin{itemize}
 	 \item Atributos
 	 \item Entidades
 	 \item Relaci\'on de vecindad
 	 \item Abstracci\'on estandar
 	 \item Derivaci\'on y herencia
 	 \item Relaciones anidadas
 	\end{itemize}
 \end{itemize}
\end{itemize}
\end{frame}

\subsection{Integridad de Datos}


\subsubsection{Separaci\'on entre el esquema y la instancia}
Las bases de datos orientadas a grafos pueden tomar tres niveles de estructuramiento de los datos:
\begin{itemize}
 \item Estructuradas: Se define un esquema y toda instancia debe respetar ese esquema de tal manera que toda entidad tenga todos los atributos que el
 esquema indica y s\'olo se puede relacionar con las entidades que el esquema especifica.
 \item Semi estructuradas: Se define un esquema pero las instancias lo respetan parcialmente. Una entidad en la instancia podr\'ia tener s\'olo algunos
 de los atributos declarados en el esquema, pero no puede tener atributos que no est\'en en el esquema. En cuanto a las relaciones, s\'olo puede usar las
 que se definieron en el esquema.
 \item No estructuradas: No se define un esquema. La instancia puede adoptar cualquier forma. La estructura de la base es din\'amica. Ej: Neo4j.
\end{itemize}
Seg\'un qu\'e nivel se implemente, la consistencia entre el esquema y la instancia puede ser m\'as o menos laxa.


\subsubsection{Identidad de Objetos e Integridad Referencial}
Para la identidad de los Objetos, en las bases de datos orientadas a grafos se utilizan 2 tipos de identificadores
\begin{itemize}
 \item Identificaci\'on seg\'un el valor de los atributos de cada Entidad.
 \item Un identificador \'unico asignado a cada Entidad, de manera independiente de sus atributos.
\end{itemize}

Tener ambas formas de identificaci\'on trae las siguientes ventajas
\begin{itemize}
 \item Capacidad de identificar a cualquier entidad, sin importar cu\'an compleja puede ser.
 \item Los objetos pueden copartir subobjetos en com\'un
 \item Simplificaci\'on en queries de b\'usqueda y actualizaci\'on
\end{itemize}


En cuanto a la Integridad Referencia, en el Modelo de Hipernodos se definen 2 tipos de restricciones
\begin{itemize}
 \item {\bf Integridad de Entidades}. Cada nodo es identificado un\'ivocamente.
 \item {\bf Integridad Referencial}. S\'olo las entidades existentes pueden ser referenciadas.
\end{itemize}


\subsubsection{Depedencia funcional}

Al igual que en el Modelo de Bases de Datos Relacionales, se intent\'o traer el concepto de Dependencia Funcional. Esto no fu\'e tan facil porque no en todos los modelos de Base de Datos orientadas a Grafos este concepto pod\'ia ser presentado de manera sencilla.

Por el momento el concepto existe para el modelo de Hipernodos. Una Dependencia Funcional $A$ $\rightarrow$ $B$, en donde tanto $A$ como $B$ son conjuntos de atributos, denota que para cada Hipernodo de la Base de Datos, $A$ determina el valor de $B$.



\begin{comment}

\section{Modelo Teórico}

Schema vs Data instances
Representación de entidades / relaciones (pag 12-13)

* Restricciones que aseguran integridad: consistencia entre esquema e instancia. El grafo asegura que no puede haber foreing keys apuntando a nada.
A veces los nodos tienen IDs. Asignar ``tipos'' a las entidades y a las relaciones (con el schema). Datos estructurados, semi estructurados, no estructurados.
Declarar dependencias funcionales explicitamente.

\section{Lenguajes de Consulta y Manipulación}

Operaciones de teoría de grafos. Pattern Matching. Encontrar el shortest path. Neighborhood. Paths. 
Leguajes graficos (pag18), de texto (cypher), 

Cypher- http://neo4j.com/developer/cypher-query-language/
http://neo4j.com/docs/stable/cypherdoc-finding-paths.html
http://neo4j.com/developer/cypher/
\end{comment}

\section{Implementaciones}
El modelos de base de datos orientadas a grafos posee muchas implementaciones diferentes. Cada una de ellas define su propio lenguaje de consulta, su estructura de datos y sus reglas de integridad. A continuación, definiremos dos implementaciones las cuales muestran la esencia del modelo.

\begin{itemize}
\item Modelo de Hypernodos: Es una implementación la cual fue desarrollada en 1990 y posee una estructura de datos que está basada en hypernodos. Un hypernodo es un grafo dirigido cuyos nodos pueden ser tanto otros grafos, como nodos primitivos (strings, enteros, etc.). De esta forma, se permite que exista el anidamiento de grafos, lo cual posibilita el encapsulamiento de la información. La ventaja de los hypernodos es que pueden ser usados para representar objetos arbitrariamente complejos.

En 1994 se introdujo su esquema y un chequeo de tipos el cual fue representado con el anidamiento de grafos. Si nos fijamos en el ejemplo de la figura \ref{hypernodo}, podemos notar que el esquema define a una $persona$ como un objeto complejo el cual posee las propiedades $nombre$ y $apellido$ del tipo string. Además, define de manera recursiva a la propiedad $padre$. Si miramos su instancia, veremos como se relacionan las diferentes instancias de $persona$ a través de su propiedad $padre$

\begin{center}
  
  	\includegraphics[width=0.6\textwidth]{img/ej4_hypernodo.png}\label{hypernodo}
 
\end{center}

El modelo de Hypernodos tiene un lenguaje de consulta basado en reglas llamado $Hyperlog$. Éste es un lenguaje declarativo basado en grafos el cual soporta tanto las queries como los updates del modelo de hypernodos. Hyperlog fue pensado para usuarios no expertos en base de datos, es por eso que que cuenta con un pequeño número de construcciones gráficas las cuales permiten definir, por ejemplo, un esquema a través de una interfaz gráfica. 

En 1995 se agregó el concepto de dependencia funcional entre hypernodos. Dicha dependencia se denota como $A$ $\rightarrow$ $B$, en donde $A$ y $B$ son un conjunto de atributos y $A$ determina el valor de $B$ en todos los hypernodos de la base de datos. Hyperlog denomina a las dependencias funcionales como programas y es lo que le permite realizar queries y updates. 

\item Neo4j: Es una base de datos orientada a grafos que fue implementada en Java. Actualmente cuenta con dos tipos de licencia, una comercial y una $Affero\ General\ Public\ License$ (AGPL). 

La base de Neo4j utiliza $grafos\ de\ propiedad$ \footnote{Es un grafo con peso y etiquetas que permite asignar propiedades tanto en el nodo como en las aristas} para poder extraer datos de forma ágil y sencilla. Además de esto, dicho grafo permite una gran flexibilidad para representar estructuras complejas y permite que el sistema sea escalable dado que la inserción de nodos y aristas nuevas no afectan al sistema anterior.

Si observamos su rendimiento, podemos notar que es mucho mejor que las bases de datos relacionales y las no relacionales. Esto se debe a que al aumentar la dificultad la consulta, las bases de datos orientadas a grafos solo actualizan los nodos y las relaciones de la búsqueda, lo cual, hace que el proceso sea mucho más óptimo ya que no se modifica el grafo completo.

Actualmente neo4j es utilizado por empresas como eBay (el cual la usa para planificar las rutas del servicio de correo electrónico)  y Walmart (que analiza las ventas de los productos para poder recomendarle a sus clientes que comprar). Dichas empresas lo usan dado que las bases de datos orientadas a grafos les permite realizar consultas sobre la relación de los diferentes objetos de manera óptima.


\end{itemize}

Por otro lado, existen implementaciones las cuales no utilizan el modelo en su totalidad, sino que utilizan a los grafos para navegar, definir vistas o como representaciones de lenguajes. Un ejemplo de esto es la $Database\ graph\ view$ la cual provee una definición funcional de grafos sobre los datos que pueden estar almacenados en una base de datos ya sea relacional, orientada a objetos o en un file system. Como se puede observar, el modelo propone una visión de grafos en los datos para poder manejarlos y hacer consultas sobre ellos sin importar como está implementada la base de datos.


\newpage
\section{Bibliografía}
\begin{description}
 \item \emph{Survey of Graph Database Models}, Renzo Angles, Claudio Gutierrez, Universidad de Chile. ACM Computing Surveys, Vol. 40, No. 1, Article 1,
 2008.
 \item \url{http://neo4j.com/developer/graph-db-vs-rdbms/}
\end{description}

\end{document}

