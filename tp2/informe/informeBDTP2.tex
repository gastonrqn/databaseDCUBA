\documentclass[a4paper, 10pt, twoside]{article}

\usepackage[top=1in, bottom=1in, left=1in, right=1in]{geometry}
\usepackage[utf8]{inputenc}
\usepackage[spanish, es-ucroman, es-noquoting]{babel}
\usepackage{setspace}
\usepackage{fancyhdr}
\usepackage{lastpage}
\usepackage{amsmath}
\usepackage{amsfonts}
\usepackage{amsthm}
\usepackage{verbatim}
\usepackage{graphicx}
\usepackage{float}
\usepackage{enumitem} % Provee macro \setlist
\usepackage{tabularx}
\usepackage{multirow}
\usepackage{hyperref}
\usepackage{multicol}
\usepackage[toc, page]{appendix}
\usepackage{ulem} % para subrayados especiales
\usepackage{color}
\usepackage{framed}
\usepackage{ifthen}

\definecolor{shadecolor}{rgb}{0.95,0.95,0.95}


%%%%%%%%%% Configuración de Fancyhdr - Inicio %%%%%%%%%%
\pagestyle{fancy}
\thispagestyle{fancy}
\lhead{Trabajo Práctico 2 · Bases de Datos}
\rhead{Heredia · Requeni · Vega · Vita}
\renewcommand{\footrulewidth}{0.4pt}
\cfoot{\thepage /\pageref{LastPage}}

\fancypagestyle{caratula} {
   \fancyhf{}
   %\cfoot{\thepage /\pageref{LastPage}}
   \renewcommand{\headrulewidth}{0pt}
   \renewcommand{\footrulewidth}{0pt}
}
%%%%%%%%%% Configuración de Fancyhdr - Fin %%%%%%%%%%


%%%%%%%%%% Configuración de Hyperref - Inicio %%%%%%%%%%
\hypersetup{%
 % Para que el PDF se abra a pagina completa.
  pdfstartview= {FitH \hypercalcbp{\paperheight-\topmargin-1in-\headheight}},
  pdfauthor={Heredia,Requeni,Vega,Vita},
  pdfsubject={BD-TP2},
 %pdfkeywords={keyword1} {key2} {key3},
 colorlinks=true,
  linkcolor=black,
  urlcolor=blue
}
%%%%%%%%%% Configuración de Hyperref - Fin %%%%%%%%%%


%%%%%%%%%% Miscelánea - Inicio %%%%%%%%%%
% Evita que el documento se estire verticalmente para ocupar el espacio vacío
% en cada página.
\raggedbottom

% Deshabilita sangría en la primer línea de un párrafo.
\setlength{\parindent}{0em}

% Separación entre párrafos.
\setlength{\parskip}{0.5em}

% Separación entre elementos de listas.
\setlist{itemsep=0.5em}

% Asigna la traducción de la palabra 'Appendices'.
\renewcommand{\appendixtocname}{Apéndices}
\renewcommand{\appendixpagename}{Apéndices}
%%%%%%%%%% Miscelánea - Fin %%%%%%%%%%


\begin{document}
\pagenumbering{gobble} 
 \normalem


%%%%%%%%%%%%%%%%%%%%%%%%%%%%%%%%%%%%%%%%%%%%%%%%%%%%%%%%%%%%%%%%%%%%%%%%%%%%%%%
%% Carátula                                                                  %%
%%%%%%%%%%%%%%%%%%%%%%%%%%%%%%%%%%%%%%%%%%%%%%%%%%%%%%%%%%%%%%%%%%%%%%%%%%%%%%%


\thispagestyle{caratula}

\begin{center}

\includegraphics[height=2cm]{caratula/DC.png} 
\hfill
\includegraphics[height=2cm]{caratula/UBA.jpg} 

\vspace{2cm}

Departamento de Computación,\\
Facultad de Ciencias Exactas y Naturales,\\
Universidad de Buenos Aires

\vspace{4cm}

\begin{Huge}
Trabajo Práctico 2
\end{Huge}

\vspace{0.5cm}

\begin{huge}
Bases de Datos Orientadas a Grafos
\end{huge}

\vspace{0.5cm}

\begin{Large}
Bases de Datos
\end{Large}

\vspace{1cm}

Primer Cuatrimestre de 2015

\vspace{3.5cm}

\begin{Large}
Grupo 4
\end{Large}

\vspace{0.5cm}

\begin{tabular}{|c|c|c|}
\hline
Apellido y Nombre & LU & E-mail\\
\hline
Heredia, Martín  & 146/11 & {\tt martin.herediaf@gmail.com}\\
Requeni, Gastón  & 400/11 & {\tt grequeni@hotmail.com}\\
Vega, Leandro    & 698/11 & {\tt leandrogvega@gmail.com}\\
Vita, Sebastián  & 149/11 & {\tt sebastian\_vita@yahoo.com.ar}\\
\hline
\end{tabular}

\end{center}

\newpage
\pagenumbering{arabic} 

%%%%%%%%%%%%%%%%%%%%%%%%%%%%%%%%%%%%%%%%%%%%%%%%%%%%%%%%%%%%%%%%%%%%%%%%%%%%%%%
%% Índice                                                                    %%
%%%%%%%%%%%%%%%%%%%%%%%%%%%%%%%%%%%%%%%%%%%%%%%%%%%%%%%%%%%%%%%%%%%%%%%%%%%%%%%


\tableofcontents

\newpage

\section{Introducción}

¿Qué es?

\section{Motivación}

Interconectividad de datos, topología.

Hacer los Joins (pag 5)

Es similar a implementar el DER directamente (el mismo DER de siempre).

http://neo4j.com/developer/graph-db-vs-rdbms/


\section{Modelo Teórico}

Schema vs Data instances
Representación de entidades / relaciones (pag 12-13)

* Restricciones que aseguran integridad: consistencia entre esquema e instancia. El grafo asegura que no puede haber foreing keys apuntando a nada.
A veces los nodos tienen IDs. Asignar ``tipos'' a las entidades y a las relaciones (con el schema). Datos estructurados, semi estructurados, no estructurados.
Declarar dependencias funcionales explicitamente.

\section{Lenguajes de Consulta y Manipulación}

Operaciones de teoría de grafos. Pattern Matching. Encontrar el shortest path. Neighborhood. Paths. 
Leguajes graficos (pag18), de texto (cypher), 

Cypher- http://neo4j.com/developer/cypher-query-language/
http://neo4j.com/docs/stable/cypherdoc-finding-paths.html
http://neo4j.com/developer/cypher/

\section{Implementaciones}
Hypernode Model, neo4j


\end{document}

