\begin{itemize}
	\item Si un ciudadano es fiscal de una mesa en una elección, debe pertenecer a un partido político en esa misma elección.
	\item Los fiscales de una misma mesa son de distintos partidos políticos.
	\item Si un ciudadano es candidato en una elección, debe pertenecer a un partido político en esa misma elección.
	\item Si un ciudadano es fiscal y candidato en una misma elección, entonces en ambas relaciones debe estar asociado al mismo partido político.
	\item Si un ciudadano es ganador en una elección, debe ser candidato en la misma elección. Ganador debe ser el candidato con más cantVotos de esa elección
	\item Una elección cumple que el atributo “fechaFin” es menor a la fecha actual si y sólo si tiene un ganador (ciudadano) asociado.
	\item Una Mesa Electoral no puede emitir votos después de la “fechaFin” ni antes de la “fechaInicio” de la elección a la que corresponde.
	\item La elección tiene mesas que pertenecen a un centro de votación que a su vez se encuentra en un municipio. Este municipio debe ser alguno de los municipios que abarca la elección o debe pertenecer a alguna de las provincias que abarca la elección.
	\item La elección debe abarcar al menos un municipio o una provincia.
	\item Para cada elección, todo ciudadano es o bien fiscal, o bien presidente, o bien vicepresidente, o bien técnico, o ninguna de las anteriores, de una única mesa (en esa elección).
	\item Si una elección abarca al menos un Municipio, entonces no abarca a ninguna Provincia, y viceversa.
	\item Dado un Centro de Votación, que tiene asociado un Municipio, todas las elecciones en las cuales participan alguna de sus mesas tienen que abarcar a este Municipio, ya sea porque las elecciones abarcan al mismo, o porque abarcan a una provincia la cual el Municipio pertenezca.
	\item Dada una elección en un municipio, todas las mesas de dicha elección deben contener a este Municipio.
	\item Una mesa emite voto a un ciudadano que es candidato en esa elección.
	\item La fecha en la que vota un ciudadano en una elección está entre la “fechaInicio” y “fechaFin” de la misma.
	\item Cada MESA tiene 1 PRESIDENTE DE MESA, 1 VICEPRESIDENTE DE MESA, 1 TECNICO y varios fiscales.
	\item La cantidad de votos obtenidos en una votacion por un CANDIDATO ELECTORAL o una OPCION ELECTORAL siempre es igual a la suma de la cantidad de votos que obtuvo en todas las MESAS ELECTORALES de dicha votacion
\end{itemize}