\begin{itemize}
	\item {\bf De las interrelaciones}
	\begin{itemize}
		\item Los fiscales de una misma mesa son de distintos partidos políticos.
		\item Si un ciudadano es fiscal y candidato en una misma elección, entonces en ambas relaciones debe estar asociado al mismo partido político.
		\item Una votación se relaciona con al menos un municipio ó una provincia, pero no puede estar relacionado con un municipio y una provincia
		a la vez (es decir que los escenarios posibles son: votación que abarca de uno a muchos municipios o votación que abarca de una a muchas
		provincias, pero no puede abarcar municipios y provincias).
		\item Para cada elección, todo ciudadano es o bien fiscal, o bien presidente, o bien vicepresidente, o bien técnico, o ninguna de las anteriores, de una única mesa (en esa elección) (es decir que no puede haber un ciudadano que cumpla dos roles de esos en la misma votación).
		\item Dada una votación, participa en la relación VotóEn sólo las mesas que pertenecen a centros de votación cuyo municipio (o la pronvincia
		de éste) es abarcado por la votación.
		\item Un candidato o una opción electoral sólo pueden ser votados en mesas que pertenecen a algún centro de votación cuyo municipio (o la
		provincia de éste) es abarcado por la votación.
		\item Cada mesa tiene exactamente un presidente, un vicepresidente, y cero o más fiscales.
		\item Si un ciudadano tiene fechaDeDefunción distinto de null, entonces no puede participar en ninguna relación asociada a una votación
		con fechaInicio mayor a fechaDeDefunción.
		\item Un ciudadano no puede participar en ninguna relación asociada a una votación tal que fechaInicio sea anterior a fechaIngresoPadrón del
		ciudadano.
	\end{itemize}
	
	\item {\bf De los atributos}
	\begin{itemize}
		\item La relación VotóEn no puede tener el atributo fecha mayor a fechaFin ni menor a fechaInicio de la votación a la que corresponde.
		\item Una elección cumple que el atributo fechaFin es menor a fechaActual y ninguno de ellos es nulo.
		\item La cantidad de votos de un candidato electoral debe ser igual a la suma de la cantidad de votos en cada tupla de la relación esVotado
		(asociadas a ese candidato).
		\item La cantidad de votos de una opción electoral debe ser igual a la suma de la cantidad de votos en cada tupla de la relación esVotada
		(asociadas a esa opción electoral).
		\item Los atributos cantVotos de las entidades opción electoral y candidato electoral, y de las interrelaciones esVotado y esVotada, son todos
		no nulos y mayores o iguales que cero.
		\item El atributo fechaIngresoPadrón de un ciudadano debe ser mayor que fechaDeNacimiento y menor que fechaDeDefunción.
	\end{itemize}
\end{itemize}
